\documentclass{article}
\usepackage[spanish]{babel}
\usepackage{mathtools}
\usepackage{amssymb}
\usepackage{amsthm}

% Math Symbols
\newcommand{\realNumbers}{\mathbb{R}}
\newcommand{\complexNumbers}{\mathbb{C}}
\newcommand{\upperHalfPlane}{\mathcal{H}}

% Math Paired Delimiters
\DeclarePairedDelimiter{\norm}{\lVert}{\rVert}

% Math intervals %
%%%%%%%%%%%%%%%%%%
%\begin{}
\usepackage{etoolbox}
\newcommand{\leftOpenInterval}{\left]}
\newcommand{\rightOpenInterval}{\right[}
\newcommand{\leftClosedInterval}{\left[}
\newcommand{\rightClosedInterval}{\right]}
\newcommand{\interval}[3]{%
  \ifstrequal{#1}{oo}{%
    \leftOpenInterval #2, #3 \rightOpenInterval%
  }{%
    \ifstrequal{#1}{co}{%
      \leftClosedInterval #2, #3 \rightOpenInterval%
    }{%
      \ifstrequal{#1}{oc}{
        \leftOpenInterval #2, #3 \rightClosedInterval%
      }{%
        \ifstrequal{#1}{cc}{
          \leftClosedInterval #2, #3 \rightClosedInterval
        }{%
        }%
      }%
    }%
  }%
}
%\end{}
% End math intervals %
%%%%%%%%%%%%%%%%%%%%%%

% Theorem Environments
%\begin{}
\newtheorem{theorem}{Teorema}
\newtheorem{proposition}{Proposición}
\newtheorem{problem}{Problema}
\newtheorem{definition}{Definición}
\theoremstyle{remark}
\newtheorem{remark}{Observación}
%\end{}

% Unit Vectors
%\begin{}
\usepackage{bm}
\newcommand{\uveci}{{\bm{\hat{\textnormal{\bfseries\i}}}}}
\newcommand{\uvecj}{{\bm{\hat{\textnormal{\bfseries\j}}}}}
\newcommand{\uvecn}{{\bm{\hat{\textnormal{\bfseries n}}}}}
%\end{}

\title{Flujo ideal alrededor de un cilindro: potencial complejo}
\author{Pablo Brianese}
\begin{document}
  \maketitle
  Para el caso particular de un flujo irrotacional y no viscoso, la velocidad del fluído puede describirse (asumimos un sistema efectivamente bidimensional)
  \begin{align}
    \vec{v} = \vec{\nabla} \phi
  \end{align}
  donde \(\phi\) es el potencial de velocidades.
  Si además el flujo es incompresible (\(\vec{\nabla} \cdot \vec{v} = 0\)), resulta
  \begin{align}
    \vec{\nabla}^2 \phi
    =
    \frac{\partial^2 \phi}{\partial x^2}
    +
    \frac{\partial^2 \phi}{\partial y^2}
    =
    0
  \end{align}

  En mecánica de fluídos se demuestra que, para que la condición de irrotacionalidad \(\vec{v} \times \vec{v} = 0\) se mantenga en el tiempo, es necesario que el fluido sea noviscoso.

  Dado el portencial de velocidades, podemos formar con su armónica conjugada \(\psi\) la función analítica \(f(z) = \phi(x, y) + i \psi(x, y)\).
  Las líneas \(\{\psi = \text{const}\}\) son trayectorias ortogonales a las líneas equipotenciales y son llamadas \emph{líneas de corriente}.
  Es fácil ver que
  \begin{align}
    \vec{v}
    &=
    \vec{\nabla} \phi
    \\
    &=
    \left( \frac{\partial \phi}{\partial x}, \frac{\partial \phi}{\partial y} \right)
    \\
    &=
    \left( \frac{\partial \phi}{\partial x}, \frac{\partial \phi}{\partial y} \right)
    \qquad
    \text{por Cauchy--Riemann}
    \\
    &=
    \overline{\frac{\partial f}{\partial x}}
    \\
    &=
    \overline{f'(z)}
  \end{align}
  Es usual definir la \emph{velocidad compleja} \(q = \overline{f'}\).
  Con lo cual \(\norm{\vec{v}} = \norm*{\overline{f'}} = \norm{f'}\).

  Para el flujo irrotacional alrededor de un cilindro, el potencial complejo es \(f(z) = U_0 (z + R^2 / z)\), donde \(R\) es el radio del cilindro y \(U_0\) es la velocidad en el inlet.
  En coordenadas reales el potencial complejo es
  \begin{align}
    f(x, y)
    &=
    U_0 \left( (x + i y) + R^2 / (x + i y) \right)
    \\
    &=
    U_0 \left( (x + i y) + R^2 \frac{x - i y}{x^2 + y^2} \right)
    \\
    &=
    U_0 \left( (x + i y) + \frac{R^2}{x^2 + y^2} (x - i y) \right)
    \\
    &=
    U_0 \left( \left( 1 + \frac{R^2}{x^2 + y^2} \right) x + i \left( 1 - \frac{R^2}{x^2 + y^2} \right) y \right)
    \\
    &=
    \left(
      U_0 \left( 1 + \frac{R^2}{x^2 + y^2} \right) x,
      U_0 \left( 1 - \frac{R^2}{x^2 + y^2} \right) y
    \right)
  \end{align}
  y su derivada es
  \begin{align}
    f'(x, y)
    &=
    \left. \frac{d}{d z} U_0 \left( z + \frac{R^2}{z} \right) \right\vert_{z = x + i y}
    \\
    &=
    U_0 \left( 1 - \frac{R^2}{(x + i y)^2} \right)
    \\
    &=
    U_0 \left( 1 - \frac{R^2}{x^2 - y^2 + i 2 x y} \right)
    \\
    &=
    U_0 \left( 1 - \frac{R^2}{(x^2 - y^2)^2 + 4 x^2 y^2} (x^2 - y^2 - i 2 x y) \right)
    \\
    &=
    U_0 \left( 1 - \frac{R^2}{x^4 + 2 x^2 y^2 + y^4} (x^2 - y^2 - i 2 x y) \right)
    \\
    &=
    U_0 \left( 1 - \frac{R^2}{(x^2 + y^2)^2} (x^2 - y^2 - i 2 x y) \right)
    \\
    &=
    U_0 \left( 1 - \frac{R^2}{(x^2 + y^2)^2} (x^2 - y^2) + i \frac{R^2}{(x^2 + y^2)^2} (2 x y) \right)
    \\
    &=
    \left(
      U_0 \left( 1 - \frac{R^2}{(x^2 + y^2)^2} (x^2 - y^2) \right),
      U_0 \frac{R^2}{(x^2 + y^2)^2} (2 x y)
    \right)
  \end{align}
  Por lo tanto, el potencial de velocidades \(\phi\), la función de corriete \(\psi\), y el campo de velocidades \(\vec{v}\) son
  \begin{align}
    \phi(x, y)
    &=
    U_0 \left( 1 + \frac{R^2}{x^2 + y^2} \right) x \\
    \psi(x, y)
    &=
    U_0 \left( 1 - \frac{R^2}{x^2 + y^2} \right) y
    \\
    \vec{v}(x, y)
    &=
    \left(
      U_0 \left( 1 - \frac{R^2}{(x^2 + y^2)^2} (x^2 - y^2) \right),
      U_0 \frac{R^2}{(x^2 + y^2)^2} (- 2 x y)
    \right)
  \end{align}
  
\end{document}