\documentclass{article}
\usepackage[spanish]{babel}
\usepackage{etoolbox}
\usepackage{mathtools}
\usepackage{amssymb}
\usepackage{hyperref}

\newcommand{\realNumbers}{\mathbb{R}}

\DeclarePairedDelimiter\abs{\lvert}{\rvert}%
\DeclarePairedDelimiter\norm{\lVert}{\rVert}%

% Math intervals %
%%%%%%%%%%%%%%%%%%
\newcommand{\leftOpenInterval}{\left]}
\newcommand{\rightOpenInterval}{\right[}
\newcommand{\leftClosedInterval}{\left[}
\newcommand{\rightClosedInterval}{\right]}
\newcommand{\interval}[3]{%
  \ifstrequal{#1}{oo}{%
    \leftOpenInterval #2, #3 \rightOpenInterval%
  }{%
    \ifstrequal{#1}{co}{%
      \leftClosedInterval #2, #3 \rightOpenInterval%
    }{%
      \ifstrequal{#1}{oc}{
        \leftOpenInterval #2, #3 \rightClosedInterval%
      }{%
        \ifstrequal{#1}{cc}{
          \leftClosedInterval #2, #3 \rightClosedInterval
        }{%
        }%
      }%
    }%
  }%
}
% End math intervals %
%%%%%%%%%%%%%%%%%%%%%%


\title{Ecuación de Transporte}
\begin{document}
  \maketitle
  La ecuación de transporte se lee
  \begin{align}
    \frac{\partial T}{\partial t} = \nabla (D \nabla T) - \nabla \cdot (\vec{v} \cdot T) + S
  \end{align}
  \begin{itemize}
    \item \(T\) es la variable de interés (concentración de especie para el transporte de masa, temperatura para el transporte de calor);
    \item \(D\) es el coeficiente de difusión, este puede ser propio de la especie en el caso del movimiento de partículas o referirse a la difusividad térmica en el caso del transporte de calor;
    \item \(\vec{v}\) es el campo de velocidades con que se mueve la cantidad.
    Es una función del tiempo y del espacio.
    Por ejemplo, en el caso de la advección
    \footnote{Definición de advección}
    ,
    \(T\) podría ser la concentración de sal en un río, y luego \(\vec{v}\) sería la velocidad del flujo de agua como función del tiempo y la ubicación.
    En otro ejemplo, \(T\) podría ser la concentración de pequeñas burbujas en un lago calmo, y luego \(\vec{v}\) sería la velocidad de las burbujas subiendo hacia la superficie por flotación dependiendo del tiempo y de la ubicación de la burbuja.
    \item \(S\) describe fuentes o sumideros para la cantidad \(T\).
    \item Por ejemplo, para una especie química, \(S > 0\) quiere decir que una reacción química está creando más sustancia de esta especie, y \(S < 0\) quiere decir que una reacción química está destruyendo la especie.
    Para el transporte de calor, \(R > 0\) podría ocurrir si se genera energía térmica mediante fricción.
  \end{itemize}
  
  El lado derecho de la ecuación es la suma de tres contribuciones.
  \begin{itemize}
    \item La primera, \(\nabla \cdot (D \nabla T)\), describe la difusión.
    Imaginemos que \(T\) es la concentración de un químico.
    Cuando la concentración es baja en un lugar comparada con el área que lo rodea (es decir, se trata de un mínimo local de la concentración), la sustancia se difundirá desde los alrededores hacia en interior de dicho lugar, entonces la concentración aumentará.
    De forma recíproca, si en este lugar la concentración es alta en comparación con los alrededores (es decir, se trata de un maximo local de la concentración), entonces la substancia se difundirá hacia el exterior de dicho lugar y la concentración decrecerá.
    La difusión neta es proporcional al Laplaciano de la concentración si la difusividad \(D\) es constante.
    \item La segunda contribución, \(- \nabla \cdot (\vec{v} \cdot T)\), describe la convección (o advección).
    Imaginemos estar parados sobre el banco de un río, midiendo la salinidad del agua (cantidad de sal) cada segundo.
    Río arriba, alguien arroja un balde de sal dentro del río.
    Al rato, veríamos aumentar la salinidad súbitamente, luego caer, mientras pasa la zona de agua salada.
    Por lo tanto, la concentración en una ubicación dada puede cambiar por el flujo.
    \item La contribución final, \(S\), describe la creación o destrucción de una cantidad.
    Por ejemplo, si \(T\) es la concentración de una molécula, entonces \(S\) describe como la molécula puede ser creada o destruída por reacciones químicas.
    \(S\) puede ser una función de \(T\) o de otros parámetros.
  \end{itemize}

    En nuestro caso, tanto \(\vec{v}\) como \(S\) son nulos.
    Además \(D\) es constante en el tiempo y no varía en el espacio.
    Luego, debemos resolver la ecuación
    \begin{align}
      \frac{\partial T}{\partial t} = D \Delta T
    \end{align}
    Esta es la ecuación del calor.
    Para resolverla, seguimos una técnica de resolución que fue descubierta por Joseph Fourier y publicada en \emph{Théorie analytique de la chaleur} el año 1822.
    La ecuación es
    \begin{align}
      \label{equation:diffusion}
      \frac{\partial T}{\partial t}
      =
      D \frac{\partial^2 T}{\partial x^2} + D \frac{\partial^2 T}{\partial y^2}
    \end{align}
    Donde \(T = T(x, y, t)\) es una función de tres variables \(x\), \(y\), \(t\).
    Aquí \(x\), \(y\) son las variables espaciales, de modo tal que \(x \in \interval{cc}{0}{L_x}\), \(y \in \interval{cc}{0}{L_y}\);
    y \(t\) es la variable temporal, de modo tal que \(t \geq 0\).
    Asumimos la condición inicial
    \begin{align}
      \label{equation:initialCondition}
      T(x, y, 0)
      =
      \left\{
        \begin{aligned}
          1 &&\text{si } x \leq L_x /2 \\
          0 &&\text{si } x > L_x / 2
        \end{aligned}
      \right.
    \end{align}
    y las condiciones de borde para todo \(x \in \interval{cc}{0}{L_x}\), \(y \in \interval{cc}{0}{L_y}\), \(t > 0\)
    \begin{align}
      \label{systemOfEquations:initialConditions}
      T(0, y , t)
      =
      T(x, 0, t)
      =
      T(L_x, y, t)
      =
      T(x, L_y, t)
      =
      0
    \end{align}
    Buscaremos una solución a la ecuación \eqref{equation:diffusion}, distinta de la solución identicamente nula, que satisfaga las condiciones de borde \eqref{systemOfEquations:initialConditions} pero que tenga la siguiente propiedad:
    \(T\) es un producto en el cual la dependencia de \(T\) respecto de \(x\), \(y\), \(t\) está separada.
    Esto es:
    \begin{align}
      T(x, y, t)
      =
      X(x) Y(y) \tau(t)
    \end{align}
    Esta técnica de resolución recibe el nombre de separación de variables.
    Substituyendo esta fórmula para \(T\) en las condiciones iniciales \eqref{systemOfEquations:initialConditions}
    se deduce
    \begin{align}
      &X(0) Y(y) \tau(t)
      =
      X(x) Y(0) \tau(t)
      =
      X(L_x) Y(y) \tau(t)
      =
      X(x) Y(L_y) \tau(t)
      =
      0
      \\
      \label{equation:separatedInitialConditions}
      &\Rightarrow X(0) = Y(0) = X(L_x) = Y(L_y) = 0
    \end{align}
    Substituyendo esta fórmula para \(T\) en la ecuación \eqref{equation:diffusion} se deriva
    \begin{align}
      \frac{\partial}{\partial t} X(x) Y(y) \tau(t)
      &=
      D \left(
        \frac{\partial^2}{\partial x^2} X(x) Y(y) \tau(t)
        +
        \frac{\partial^2}{\partial y^2} X(x) Y(y) \tau(t)
      \right)
      \\
      X(x) Y(y) \tau'(t)
      &=
      D \left(
        X^{\prime\prime}(x) Y(y) \tau(t)
        +
        X(x) Y^{\prime\prime}(y) \tau(t)
      \right)
      \\
      \frac{\tau'(t)}{D \tau(t)}
      &=
      \frac{X^{\prime\prime}(x) Y(y) + X(x) Y^{\prime\prime}(y)}{X(x) Y(y)}
      \\
      \label{equation:separatedDifferentialEquation}
      \frac{\tau'(t)}{D \tau(t)}
      &=
      \frac{X^{\prime\prime}(x)}{X(x)}
      +
      \frac{Y^{\prime\prime}(y)}{Y(y)}
    \end{align}
    Dado que el lado derecho de \eqref{equation:separatedDifferentialEquation} depende solo de las variables espaciales y el lado izquierdo solo del tiempo, ambos lados son iguales a un valor constante \(\lambda \in \realNumbers\).
    De este modo
    \begin{align}
      \label{equation:tauLambda}
      \frac{\tau'(t)}{D \tau(t)}
      =
      \frac{X''(x)}{X(x)} + \frac{Y''(y)}{Y(y)}
      =
      \lambda
    \end{align}
    Derivando a ambos lados de esta ecuación con respecto a \(x\), \(y\), se deduce
    \begin{align}
      \frac{d}{d x} \frac{X^{\prime\prime}(x)}{X(x)} 
      =
      \frac{d}{d y} \frac{Y^{\prime\prime}(y)}{Y(y)}
      =
      0
      &&\Rightarrow
      &&\frac{X^{\prime\prime}(x)}{X(x)} = \alpha
      &&\frac{Y^{\prime\prime}(y)}{Y(y)} = \beta
    \end{align}
    para ciertas constantes \(\alpha, \beta \in \realNumbers\).
    Resulta entonces que \(\alpha + \beta = \lambda\).

    Seguimos resolviendo la ecuación diferencial lineal de segundo orden con coeficientes constantes \(X^{\prime\prime} - \alpha X = 0\).
    Su ecuación característica es \(r^2 - \alpha = 0\), y su discriminante es \(4 \alpha\).
    Debemos considerar tres casos: \(\alpha > 0\) (el discriminante es positivo); \(\alpha = 0\) (el discriminante es nulo); \(\alpha < 0\) (el discriminante es negativo).
    En el primer caso, \(\alpha > 0\), la solución de la ecuación diferencial es de la forma \(X = C_1 e^{\sqrt{\alpha} x} + C_2 e^{- \sqrt{\alpha} x}\).
    Las condiciones iniciales \eqref{equation:separatedInitialConditions} implican que \(C_1 + C_2 = C_1 e^{\sqrt{\alpha} L_x} + C_2 e^{- \sqrt{\alpha} L_x} = 0\).
    Luego \(C_1 = C_2 = 0\).
    Podemos descartar este caso porque \(X = 0\) da lugar a una solución nula \(T = 0\) a la ecuación de transporte.
    En el segundo caso, \(\alpha = 0\), la solución a la ecuación diferencial es de la forma \(X = C_1 x + C_2\).
    Las condiciones iniciales \eqref{equation:separatedInitialConditions} implican que \(C_2 = C_1 L_x + C_2 = 0\).
    Luego \(C_1 = C_2 = 0\).
    Podemos descartar este caso porque \(X = 0\) da lugar a una solución nula \(T = 0\) a la ecuación de transporte.
    En el tercer caso, \(\alpha < 0\), la solución a la ecuación diferencial es de la forma \(X = C_1 \cos(\sqrt{\abs{\alpha}} x) + C_2 \sin(\sqrt{\abs{\alpha}} x)\).
    Las condiciones iniciales \eqref{equation:separatedInitialConditions} implican que \(C_1 = C_1 \cos(\sqrt{\abs{\alpha}} L_x) + C_2 \sin(\sqrt{\abs{\alpha}} L_x) = 0\).
    Luego \(C_2 \sin(\sqrt{\abs{\alpha}} L_x) = 0\).
    La alternativa \(C_2 = 0\) nos da nuevamente una solución nula \(X = 0\) y coherentemente \(T = 0\).
    Por otro lado, la alternativa \(\sin(\sqrt{\abs{\alpha}} L_x) = 0\) nos dá un número infinito de soluciones
    \begin{align}
      X(x) = \sin(n \pi x / L_x)
    \end{align}
    parametrizadas por un número entero \(n\) tal que \(\alpha = - n^2 \pi^2 / L_x^2\).

    Repetir este análisis para la función \(Y\), usando la ecuación \(Y^{\prime\prime} - \beta Y = 0\) y las condiciones de borde \eqref{equation:separatedInitialConditions}, nos lleva a encontrar infinitas soluciones
    \begin{align}
      Y(y) = \sin(m \pi y / L_y)
    \end{align}
    parametrizadas por un número entero \(m\) tal que \(\beta = - m^2 \pi^2 / L_y^2\).

    Finalmente, la ecuación \eqref{equation:tauLambda} se transforma, al remplazar los valores de \(\alpha\), \(\beta\)
    \begin{align}
      \tau'(t) = - \pi^2 (n^2 / L_x^2 + m^2 / L_y^2) D \tau(t)
    \end{align}
    y nos dice que una solución posible es
    \begin{align}
      \tau(t) = \exp\left( - \pi^2 (n^2 / L_x^2 + m^2 / L_y^2) D t \right)
    \end{align}

    Este análisis resuelve la ecuación de difusión en el caso especial en que la solución tiene sus dependencias separadas.
    Resulta
    \begin{align}
      T(x, y, t)
      =
      \sin\left(n \pi \frac{x}{L_x} \right)
      \sin\left(m \pi \frac{y}{L_y}\right)
      \exp\left(
        - \pi^2 \left(
          \frac{n^2}{L_x^2} + \frac{m^2}{L_y^2}
        \right)
        D t
      \right)
    \end{align}
    En general, la combinación lineal de soluciones a la ecuación \eqref{equation:diffusion} que satisfacen las condiciones de borde \eqref{systemOfEquations:initialConditions}
    también safisfacen ambas \eqref{equation:diffusion} y \eqref{systemOfEquations:initialConditions}.
    Puede probarse que la solución a las ecuaciones simultáneas
    \eqref{equation:diffusion},
    \eqref{equation:initialCondition},
    y \eqref{systemOfEquations:initialConditions},
    está dada por la serie
    \begin{align}
      T(x, y, t)
      =
      \sum_{n \geq 1, m \geq 1}
        D_{n, m}
        \sin\left( n \pi \frac{x}{L_x} \right)
        \sin\left( m \pi \frac{y}{L_y} \right)
        \exp\left(
          - \pi^2 \left(
            \frac{n^2}{L_x^2} + \frac{m^2}{L_y^2}
          \right)
          D t
        \right)
    \end{align}
    donde
    \begin{align}
      D_{n, m}
      &=
      \frac{
        \int_0^{L_x}
          \int_0^{L_y}
            T(x, y, 0)
            \sin\left( n \pi \frac{x}{L_x} \right)
            \sin\left( m \pi \frac{y}{L_y} \right)
          d y
        d x
      }{
        \int_0^{L_x}
          \int_0^{L_y}
          \sin\left( n \pi \frac{x}{L_x} \right)^2
            \sin\left( m \pi \frac{y}{L_y} \right)^2
          d y
        d x
      }
      \\
      &=
      \frac{4}{L_x L_y} 
      \int_0^{L_x}
        \int_0^{L_y}
          T(x, y, 0)
          \sin\left( n \pi \frac{x}{L_x} \right)
          \sin\left( m \pi \frac{y}{L_y} \right)
        d y
      d x
    \end{align}

    En nuestro caso, la condición inicial está dada por \eqref{equation:initialCondition}, de modo que
    \begin{align}
      D_{n, m}
      &=
      \frac{4}{L_x L_y}
      \int_0^{L_x / 2}
        \int_0^{L_y}
          \sin\left( n \pi \frac{x}{L_x} \right)
          \sin\left( m \pi \frac{y}{L_y} \right)
        d y
      d x
      \\
      &=
      \frac{4}{L_x L_y}
      \int_0^{L_x / 2}
        \sin\left( n \pi \frac{x}{L_x} \right)
      d x
        \int_0^{L_y}
          \sin\left( m \pi \frac{y}{L_y} \right)
      d y
      \\
      &=
      \frac{4}{L_x L_y}
      \left( - \frac{L_x}{n \pi} \right)
      \left( ((- 1)^m - 1) \frac{L_y}{m \pi} \right)
      \\
      &=
      \frac{4}{n m \pi^2}
      (1 - (- 1)^m)
      \\
      &=
      \left\{
        \begin{aligned}
          \frac{8}{n m \pi^2}
            &&\text{si } m \text{ es impar;}
          \\
          0
            &&\text{si } m \text{ es par.}
        \end{aligned}
      \right.
    \end{align}

\end{document}
