\documentclass{article}
\usepackage{mathtools}
\usepackage{physics}
\begin{document}
  Sea \(F(x) = e^{- 2 x} \ln(3 x)\).
  Sus derivadas son
  \begin{align}
    \frac{\dd F}{\dd x}
    &=
    \frac{\dd}{\dd x} [e^{- 2 x}] \ln(3 x)
      + e^{- 2 x} \frac{\dd}{\dd x} [\ln(3 x)]
    \\
    &=
    - 2 e^{- 2 x} \ln(3 x)
      + e^{- 2 x} \frac{1}{x}
    \\
    &=
    e^{- 2 x} \left(
      - 2  \ln(3 x)
      + \frac{1}{x} 
    \right)
  \end{align}
  \begin{align}
    \frac{\dd^2 F}{\dd x^2}
    &=
    \frac{\dd}{\dd x} \left[
      e^{- 2 x} \left(
        - 2  \ln(3 x)
        + \frac{1}{x}
      \right)
    \right]
    \\
    &=
    \frac{\dd}{\dd x} [e^{- 2 x}] \left(
        - 2 \ln(3 x)
        + \frac{1}{x}
      \right)
      + e^{- 2 x} \frac{\dd}{\dd x} \left[
        - 2 \ln(3 x)
        + \frac{1}{x}
      \right]
    \\
    &=
    - 2 e^{- 2 x} \left(
        - 2 \ln(3 x)
        + \frac{1}{x}
      \right)
      + e^{- 2 x} \left(
        - 2 \frac{1}{x}
        - \frac{1}{x^2}
      \right)
    \\
    &=
    e^{- 2 x} \left(
      4 \ln(3 x)
      -4 \frac{1}{x}
      - \frac{1}{x^2}
    \right)
  \end{align}
  \begin{align}
    \frac{\dd^3 F}{\dd x^3}
    &=
    \frac{\dd}{\dd x} \left[
      e^{- 2 x} \left(
        4 \ln(3 x)
        -4 \frac{1}{x}
        - \frac{1}{x^2}
      \right)  
    \right]
    \\
    &=
    \frac{\dd}{\dd x} [e^{- 2 x}] \left(
      4 \ln(3 x)
      - 4 \frac{1}{x}
      - \frac{1}{x^2}
    \right)
      + e^{- 2 x} \frac{\dd}{\dd x} \left[
        4 \ln(3 x)
        - 4 \frac{1}{x}
        - \frac{1}{x^2}
      \right]
    \\
    &=
    - 2 e^{- 2 x} \left(
      4 \ln(3 x)
      - 4 \frac{1}{x}
      - \frac{1}{x^2}
    \right)
      + e^{- 2 x} \left[
        4 \frac{1}{x}
        + 4 \frac{1}{x^2}
        + 2 \frac{1}{x^3}
      \right]
    \\
    &=
    e^{- 2 x} \left(
      - 8 \ln(3 x)
      + 12 \frac{1}{x}
      + 6 \frac{1}{x^2}
      + 2 \frac{1}{x^3}
    \right)
  \end{align}
  \begin{align}
    \frac{\dd^4 F}{\dd x^4}
    &=
    \frac{\dd}{\dd x} \left[
      e^{- 2 x} \left(
        - 8 \ln(3 x)
        + 12 \frac{1}{x}
        + 6 \frac{1}{x^2}
        + 2 \frac{1}{x^3}
      \right)
    \right]
    \\
    &=
    \frac{\dd}{\dd x} [e^{- 2 x}] \left(
      - 8 \ln(3 x)
      + 12 \frac{1}{x}
      + 6 \frac{1}{x^2}
      + 2 \frac{1}{x^3}
    \right)
    \\
    &\qquad
      + e^{- 2 x} \frac{\dd}{\dd x} \left[
        - 8 \ln(3 x)
        + 12 \frac{1}{x}
        + 6 \frac{1}{x^2}
        + 2 \frac{1}{x^3}
      \right]
    \\
    &=
    - 2 e^{- 2 x} \left(
      - 8 \ln(3 x)
      + 12 \frac{1}{x}
      + 6 \frac{1}{x^2}
      + 2 \frac{1}{x^3}
    \right)
    \\
    &\qquad
      + e^{- 2 x} \left(
        - 8 \frac{1}{x}
        - 12 \frac{1}{x^2}
        - 12 \frac{1}{x^3}
        - 6 \frac{1}{x^4}
      \right)
    \\
    &=
    e^{- 2 x} \left(
      16 \ln(3 x)
      - 32 \frac{1}{x}
      - 24 \frac{1}{x^2}
      - 16 \frac{1}{x^3}
      - 6 \frac{1}{x^4}
    \right)
  \end{align}
  
  En consecuencia
  \begin{align}
    F(2) 
    = e^{- 4} \ln(6)
  \end{align}
  \begin{align}
    \frac{\dd F}{\dd x}(2) 
    = 
    e^{- 4} \left(
      - 2 \ln(6) 
      + 1/2
    \right)
    =
    \frac{1 - 4 \ln(6)}{2 e^4}
  \end{align}
  \begin{align}
    \frac{\dd^2 F}{\dd x^2}(2)
    =
    e^{- 4} \left(
      4 \ln(6) 
      - 4 \frac{1}{2} 
      - \frac{1}{2^2}
    \right)
    =
    \frac{16 \ln(6) - 9}{4 e^4}
  \end{align}
  \begin{align}
    \frac{\dd^3 F}{\dd x^3}(2)
    &=
    e^{- 4} \left(
      - 8 \ln(6)
      + 12 \frac{1}{2}
      + 6 \frac{1}{2^2}
      + 2 \frac{1}{2^3}
    \right)
    \\
    &=
    e^{- 4} \left(
      - 8 \ln(6)
      + 6 
      + 3 \frac{1}{2}
      + \frac{1}{2^2}
    \right)
    \\
    &=
    \frac{- 32 \ln(6) + 31}{4 e^4}
  \end{align}
  \begin{align}
    \frac{\dd^4 F}{\dd x^4}(2)
    &=
    e^{- 4} \left(
      16 \ln(6)
      - 32 \frac{1}{2}
      - 24 \frac{1}{2^2}
      - 16 \frac{1}{2^3}
      - 6 \frac{1}{2^4}
    \right)
    \\
    &=
    \frac{128 \ln(6) - 195}{8 e^4}
  \end{align}
  
  Así, el polinomio de Taylor, de orden 4 centrado en 2, para \(F\) es
  \begin{align}
    P(x)
    &=
    e^{- 4} \ln(6)
    + \frac{1 - 4 \ln(6)}{2 e^4} 
      (x - 2)
    \\
    &\qquad
    + \frac{16 \ln(6) - 9}{4 e^4} 
      \frac{(x - 2)^2}{2}
    + \frac{- 32 \ln(6) + 31}{4 e^4}
      \frac{(x - 2)^3}{6}
    \\
    &\qquad
    + \frac{128 \ln(6) - 195}{8 e^4}
      \frac{(x - 2)^4}{24}
    \\
    &=
    \frac{\ln(6)}{e^4}
    + \frac{1 - 4 \ln(6)}{2 e^4} 
      (x - 2)
    \\
    &\qquad
    + \frac{16 \ln(6) - 9}{8 e^4} 
      (x - 2)^2
    + \frac{- 32 \ln(6) + 31}{24 e^4}
      (x - 2)^3
    \\
    &\qquad
    + \frac{128 \ln(6) - 195}{192 e^4}
      (x - 2)^4
  \end{align}
\end{document}