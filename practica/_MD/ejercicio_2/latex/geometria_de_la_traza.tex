\documentclass{article}
\usepackage[spanish]{babel}
\usepackage{mathtools}

\DeclarePairedDelimiterX{\norm}[1]{\lVert}{\rVert}{#1}

\usepackage{bm}
\newcommand{\uvec}[1]{\boldsymbol{\hat{\textbf{#1}}}}

\title{Geometría de la traza del movimiento de un electron expuesto a un campo electromagnético externo}
\author{Pablo Brianese}
\begin{document}
  \maketitle
  En unidades del Sistema Internacional, la ecuación del movimiento para una partícula de masa \(m\), carga \(q\), expuesta a un campo electromagnético externo es
  \begin{align}
    \label{equation:movementOfChargedParticleExposedToElectromagneticField}
    m \vec{a}
    =
    q \vec{E} + q \vec{v} \times \vec{B}
  \end{align}
  Esta ecuación diferencial es lineal.
  Esto nos permite resolverla por partes.
  Sabemos que, cuando el campo elétrico es constante, la solución \(\vec{f}\) a la ecuación 
  \begin{align}
    &m \frac{d^2 \vec{f}}{d t^2} = q \vec{E}
    &&\text{está dada por}
    &&\vec{f} = \frac{q}{m} \vec{E} \frac{t^2}{2} + \left.\frac{d \vec{f}}{d t}\right|_0 t + \vec{f}_0
  \end{align}
  Si conseguimos hallar la solución \(\vec{g}\) a la ecuación
  \begin{align}
    m \frac{d^2 \vec{g}}{d t^2} = q \frac{d \vec{g}}{d t} \times \vec{B}
  \end{align}
  Entonces \(\vec{r} = \vec{f} + \vec{g}\) satisface \eqref{equation:movementOfChargedParticleExposedToElectromagneticField}.
  \begingroup
  \allowdisplaybreaks
  \begin{align}
    m \frac{d^2 \vec{g}}{d t^2}
      &=
      q \frac{d \vec{g}}{d t} \times \vec{B}
    \\
    \frac{d^2 \vec{g}}{d t^2}
      &=
      \frac{q}{m} \begin{vmatrix}
        \uvec{\i} &\uvec{\j} &\uvec{k} \\
        \frac{d g_x}{d t} &\frac{d g_y}{d t} &\frac{d g_z}{d t} \\
        B_x &B_y & B_z
      \end{vmatrix}
      \\
      &\left\{
        \begin{aligned}
          \frac{d^2 g_x}{d t^2}
            &=
            \frac{q}{m}
            \begin{vmatrix}
              \frac{d g_y}{d t} &\frac{d g_z}{d t} \\
              B_y &B_z
            \end{vmatrix}
          \\
          \frac{d^2 g_y}{d t^2}
          &=
          \frac{q}{m}
          \begin{vmatrix}
            \frac{d g_z}{d t} &\frac{d g_x}{d t} \\
            B_z &B_x
          \end{vmatrix}
          \\
          \frac{d^2 g_z}{d t^2}
          &=
          \frac{q}{m}
          \begin{vmatrix}
            \frac{d g_x}{d t} &\frac{d g_y}{d t} \\
            B_x &B_y
          \end{vmatrix}
      \end{aligned}
      \right.
    \\
    &\left\{
      \begin{aligned}
        \frac{d^2 g_x}{d t^2}
          &=
          \frac{q}{m} B_z \frac{d g_y}{d t}
          - \frac{q}{m} B_y \frac{d g_z}{d t}
        \\
        \frac{d^2 g_y}{d t^2}
          &=
          \frac{q}{m} B_x \frac{d g_z}{d t}
          - \frac{q}{m} B_z \frac{d g_x}{d t}
        \\
        \frac{d^2 g_z}{d t^2}
        &=
        \frac{q}{m} B_y \frac{d g_x}{d t}
        - \frac{q}{m} B_x \frac{d g_y}{d t}
    \end{aligned}
    \right.
    \\
    &\left\{
      \begin{aligned}
        \frac{d^2 g_x}{d t^2}
          &=
          0 \frac{d g_x}{d t}
          + \frac{q}{m} B_z \frac{d g_y}{d t}
          - \frac{q}{m} B_y \frac{d g_z}{d t}
        \\
        \frac{d^2 g_y}{d t^2}
          &=
          - \frac{q}{m} B_z \frac{d g_x}{d t}
          + 0 \frac{d g_y}{d t}
          + \frac{q}{m} B_x \frac{d g_z}{d t}
        \\
        \frac{d^2 g_z}{d t^2}
          &=
          \frac{q}{m} B_y \frac{d g_x}{d t}
          - \frac{q}{m} B_x \frac{d g_y}{d t}
          + 0 \frac{d g_z}{d t}
    \end{aligned}
    \right.
    \\
    \frac{d^2 \vec{g}}{d t^2}
    &=
    \begin{pmatrix}
      0 
      &\frac{q}{m} B_z
      &- \frac{q}{m} B_y
      \\
      - \frac{q}{m} B_z
      &0
      &\frac{q}{m} B_x
      \\
      \frac{q}{m} B_y
      &- \frac{q}{m} B_x
      &0
    \end{pmatrix}
    \frac{d \vec{g}}{d t}
    \\
    \frac{d^2 \vec{g}}{d t^2}
    &=
    A \frac{d \vec{g}}{d t}
    \\
    \frac{d \vec{g}}{d t}
    &=
    e^{A t} \left. \frac{d \vec{g}}{d t} \right|_0
    \\
    \vec{g}
    &=
    A^{- 1} e^{A t} \left. \frac{d \vec{g}}{d t} \right|_0 + \vec{g}_0
    \end{align}
  \endgroup

  \begin{align}
    A
    &=
    \begin{pmatrix}
      0 
      &\frac{q}{m} B_z
      &- \frac{q}{m} B_y
      \\
      - \frac{q}{m} B_z
      &0
      &\frac{q}{m} B_x
      \\
      \frac{q}{m} B_y
      &- \frac{q}{m} B_x
      &0
    \end{pmatrix}
  \end{align}
  autovalores de A 
  \begingroup
  \allowdisplaybreaks
  \begin{align}
    \det(\lambda - A)
    &=
    \begin{vmatrix}
      \lambda
      &- \frac{q}{m} B_z
      &\frac{q}{m} B_y
      \\
      \frac{q}{m} B_z
      &\lambda
      &- \frac{q}{m} B_x
      \\
      - \frac{q}{m} B_y
      &\frac{q}{m} B_x
      &\lambda
    \end{vmatrix}
    \\
    &=
      \lambda
      \begin{vmatrix}
        \lambda & - \frac{q}{m} B_x \\
        \frac{q}{m} B_x & \lambda
      \end{vmatrix}
    \\
      &\quad
      + \frac{q}{m} B_z
      \begin{vmatrix}
        \frac{q}{m} B_z &- \frac{q}{m} B_x \\
        - \frac{q}{m} B_y &\lambda
      \end{vmatrix}
    \\
      &\quad
      + \frac{q}{m} B_y
      \begin{vmatrix}
        \frac{q}{m} B_z &\lambda \\
        - \frac{q}{m} B_y &\frac{q}{m} B_x
      \end{vmatrix}
    \\
    &=
    \lambda (\lambda^2 + \frac{q^2}{m^2} B_x^2)
    \\
      &\quad
      + \frac{q}{m} B_z (\frac{q}{m} B_z \lambda - \frac{q^2}{m^2} B_x B_y)
      + \frac{q}{m} B_y (\frac{q^2}{m^2} B_x B_z + \frac{q}{m} B_y \lambda)
    \\
    &=
    \lambda^3 + \frac{q^2}{m^2} B_x^2 \lambda
    \\
      &\quad
      + \frac{q^2}{m^2} B_z^2 \lambda - \frac{q^3}{m^3} B_x B_y B_z
      + \frac{q^3}{m^3} B_x B_y B_z + \frac{q^2}{m^2} B_y^2 \lambda
    \\
    &=
    \lambda (\lambda^2 + \norm*{\frac{q}{m}\vec{B}}_2^2)
    \\
    &=
    \lambda (\lambda - i \norm*{\frac{q}{m}\vec{B}}_2) (\lambda + i \norm*{\frac{q}{m}\vec{B}}_2)
  \end{align}
  \endgroup
  autoespacios de A
\end{document}